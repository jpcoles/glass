\documentclass{article}

\begin{document}

\noindent[10-JAN-2011]\bigskip

\begin{itemize}

\item {\tt central\_pixel\_as\_maximum} with {\tt dgcone(20)} pushes on upper bound of {\tt t0(11,15)} prior.
\item Relaxing {\tt dgcone(20)} mildly improves sampling of $H_0^{-1}$
\item Removing {\tt central\_pixel\_as\_maximum} improves sampling of $H_0^{-1}$, but still pushes on upper bound.
\item Nearly all models produce more than four images, even though the input is a known quad.
\item Can we use lensed x-rays(?) from the observed images to build a filter for the galaxy and pick out the central image?
\item Model ensemble only has four images.
\item Using new smoothness contraint which is a function of radius. More distant pixels are less free to vary.
      \[\alpha = (s-1)(1-|r|/R) + 1\] where $0 \le |r| \le R$ is the distance of a pixel and $\alpha$ is the constraint
      coefficient that varies between $1$ and $s$.

\item
\item Use magnification of raytraced images to detect models with too many images.
\item Implement stellar mass contraint:
      \[\Phi_m = K_m \mathrm{const.}\]
\item Compare velocity dispersions
\item Create new Pgradiens which considers pixels within an arcsecond radius. Set radius from input file.

\end{itemize}


\end{document}
