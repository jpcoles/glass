\documentclass{article}

\begin{document}

\noindent[10-JAN-2011]\bigskip

Let $N \equiv 206265$ arcsec/rad and $4\pi G \equiv 1/N^2$. The speed of 
light $c$ is set to unity and our time unit is years. Length is in light-years.
Our mass unit then becomes $2.384\times 10^{31}$ kg or $11.988$ solar masses.
Positions on the sky are in arcseconds.  The Hubble time $H_0^{-1}$ is in
years.

\begin{verbatim}
Gsi := 6.673e-11 * kg^(-1) * m^3 / s^2;
ly  := 9.4605284e15 * m;
yr  := 31556926 * s;
solve(4*3.14*Gsi=206265^(-2) / marb * ly^3 / yr^2, marb);
\end{verbatim}

If we define $\nu \equiv N^2 H_0$ and the projected density $\Sigma \equiv M/L^2$ 
then $\kappa_\infty$ can be written 
\[\kappa_\infty = \frac{4\pi G}{c^2}\frac{c}{H_0}d_L\frac{M}{L^2} 
                = \frac{M}{L^2}\frac{d_L}{\nu}
                = \Sigma\frac{d_L}{\nu} \]
with the critical density $\Sigma_\mathrm{crit} = \nu/d_L$.

Light-years and radians are related as follows
\[\theta_\mathrm{rad} = \frac{L}{d_L}\frac{H_0}{c} = L\frac{\nu}{d_L N^2}\]
and in arcseconds 
\[\theta_\mathrm{arcsec} = \frac{L N}{d_L}\frac{H_0}{c} = L\frac{\nu}{d_L N}\]

The arrival time
\[\tau = \frac12(\theta-\beta)^2 c_\mathrm{LS} - \nabla^{-2}\kappa\]
is in arcsec$^2$. With the input time delays $D$ in years,
the time-delay constraint is written
\[\tau_1 - \tau_0 = \nu D / (1+z_L)\]

A density constraint from stellar mass takes the form $\kappa_\infty \geq \kappa_\mathrm{stel}$.
Population synthesis models can estimate the stellar mass, but depend on the distance to the lens
and thus also on $\nu$. The observed density from the luminosity is
\[\Sigma_\mathrm{lum} = \frac{M}{\theta^2_\mathrm{arcsec}}\left(\frac{\nu_\mathrm{true}}{d_L N}\right)^2\]
If $\Sigma_\mathrm{lum}$ is derived from a simulated galaxy, then $\nu_\mathrm{true}$ is that from the
simulation.

If $\nu$ is known then $\Sigma_\mathrm{lum} = \Sigma_\mathrm{stel}$ and the
constraint is simple: each pixel must be at least $\Sigma_\mathrm{stel} d_L /
\nu$. If $\nu$ is not known precisely then the constraint must be made weaker
to account for the scatter in $\nu_\mathrm{model}$. The constraint cannot be
tied to $\nu$ because $\nu$ does not appear linearly in $\kappa_\infty$.
Therefore, if we allow models in the range $[\nu_-, \nu_+]$ then the constraint
must be \[\kappa_\infty \geq \Sigma_\mathrm{lum}\frac{d_L}{\nu_+}\]


\end{document}
